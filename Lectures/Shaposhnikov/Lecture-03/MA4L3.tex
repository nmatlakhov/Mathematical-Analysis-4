\documentclass[12pt]{article}
\usepackage[left=1cm, right=1cm, top=2cm,bottom=1.5cm]{geometry} 

\usepackage[parfill]{parskip}
\usepackage[utf8]{inputenc}
\usepackage[T2A]{fontenc}
\usepackage[russian]{babel}
\usepackage{enumitem}
\usepackage[normalem]{ulem}
\usepackage{amsfonts, amsmath, amsthm, amssymb, mathtools,xcolor}
\usepackage{blkarray}

\usepackage{tabularx}
\usepackage{hhline}

\usepackage{accents}
\usepackage{fancyhdr}
\pagestyle{fancy}
\renewcommand{\headrulewidth}{1.5pt}
\renewcommand{\footrulewidth}{1pt}

\usepackage{graphicx}
\usepackage[figurename=Рис.]{caption}
\usepackage{subcaption}
\usepackage{float}

%%Наименование папки откуда забирать изображения
\graphicspath{ {./images/} }

%%Изменение формата для ввода доказательства
\renewcommand{\proofname}{$\square$  \nopunct}
\renewcommand\qedsymbol{$\blacksquare$}

%%Изменение отступа на таблицах
\addto\captionsrussian{%
	\renewcommand{\proofname}{$\square$ \nopunct}%
}
%% Римские цифры
\newcommand{\RN}[1]{%
	\textup{\uppercase\expandafter{\romannumeral#1}}%
}

%% Для удобства записи
\newcommand{\MR}{\mathbb{R}}
\newcommand{\MC}{\mathbb{C}}
\newcommand{\MQ}{\mathbb{Q}}
\newcommand{\MN}{\mathbb{N}}
\newcommand{\MZ}{\mathbb{Z}}
\newcommand{\MTB}{\mathbb{T}}
\newcommand{\MTI}{\mathbb{I}}
\newcommand{\MI}{\mathrm{I}}
\newcommand{\MCI}{\mathcal{I}}
\newcommand{\MJ}{\mathrm{J}}
\newcommand{\MH}{\mathrm{H}}
\newcommand{\MT}{\mathrm{T}}
\newcommand{\MU}{\mathcal{U}}
\newcommand{\MV}{\mathcal{V}}
\newcommand{\MB}{\mathcal{B}}
\newcommand{\MF}{\mathcal{F}}
\newcommand{\MW}{\mathcal{W}}
\newcommand{\ML}{\mathcal{L}}
\newcommand{\MP}{\mathcal{P}}
\newcommand{\VN}{\varnothing}
\newcommand{\VE}{\varepsilon}
\newcommand{\dx}{\, dx}
\newcommand{\dy}{\, dy}
\newcommand{\dz}{\, dz}
\newcommand{\dd}{\, d}


\theoremstyle{definition}
\newtheorem{defn}{Опр:}
\newtheorem{rem}{Rm:}
\newtheorem{prop}{Утв.}
\newtheorem{exrc}{Упр.}
\newtheorem{problem}{Задача}
\newtheorem{lemma}{Лемма}
\newtheorem{theorem}{Теорема}
\newtheorem{corollary}{Следствие}

\newenvironment{cusdefn}[1]
{\renewcommand\thedefn{#1}\defn}
{\enddefn}

\DeclareRobustCommand{\divby}{%
	\mathrel{\text{\vbox{\baselineskip.65ex\lineskiplimit0pt\hbox{.}\hbox{.}\hbox{.}}}}%
}
\DeclareRobustCommand{\ndivby}{\mkern-1mu\not\mathrel{\mkern4.5mu\divby}\mkern1mu}


%Короткий минус
\DeclareMathSymbol{\SMN}{\mathbin}{AMSa}{"39}
%Длинная шапка
\newcommand{\overbar}[1]{\mkern 1.5mu\overline{\mkern-1.5mu#1\mkern-1.5mu}\mkern 1.5mu}
%Функция знака
\DeclareMathOperator{\sgn}{sgn}

%Функция ранга
\DeclareMathOperator{\rk}{\text{rk}}
\DeclareMathOperator{\diam}{\text{diam}}


%Обозначение константы
\DeclareMathOperator{\const}{\text{const}}

\DeclareMathOperator{\codim}{\text{codim}}

\DeclareMathOperator*{\dsum}{\displaystyle\sum}
\newcommand{\ddsum}[2]{\displaystyle\sum\limits_{#1}^{#2}}
\newcommand{\ddssum}[2]{\displaystyle\smashoperator{\sum\limits_{#1}^{#2}}}
\newcommand{\ddlsum}[2]{\displaystyle\smashoperator[l]{\sum\limits_{#1}^{#2}}}
\newcommand{\ddrsum}[2]{\displaystyle\smashoperator[r]{\sum\limits_{#1}^{#2}}}

%Интеграл в большом формате
\DeclareMathOperator{\dint}{\displaystyle\int}
\newcommand{\ddint}[2]{\displaystyle\int\limits_{#1}^{#2}}
\newcommand{\ssum}[1]{\displaystyle \sum\limits_{n=1}^{\infty}{#1}_n}

\newcommand{\smallerrel}[1]{\mathrel{\mathpalette\smallerrelaux{#1}}}
\newcommand{\smallerrelaux}[2]{\raisebox{.1ex}{\scalebox{.75}{$#1#2$}}}

\newcommand{\smallin}{\smallerrel{\in}}
\newcommand{\smallnotin}{\smallerrel{\notin}}

\newcommand*{\medcap}{\mathbin{\scalebox{1.25}{\ensuremath{\cap}}}}%
\newcommand*{\medcup}{\mathbin{\scalebox{1.25}{\ensuremath{\cup}}}}%

\makeatletter
\newcommand{\vast}{\bBigg@{3.5}}
\newcommand{\Vast}{\bBigg@{5}}
\makeatother

%Промежуточное значение для sup\inf, поскольку они имеют разную высоту
\newcommand{\newsup}{\mathop{\smash{\mathrm{sup}}}}
\newcommand{\newinf}{\mathop{\mathrm{inf}\vphantom{\mathrm{sup}}}}

%Скалярное произведение
\newcommand{\inner}[2]{\left\langle #1, #2 \right\rangle }
\newcommand{\linsp}[1]{\left\langle #1 \right\rangle }
\newcommand{\linmer}[2]{\left\langle #1 \vert #2\right\rangle }

%Подпись символов снизу
\newcommand{\ubar}[1]{\underaccent{\bar}{#1}}

%%Шапка для букв сверху
\newcommand{\wte}[1]{\widetilde{#1}}
\newcommand{\wht}[1]{\widehat{#1}}
\newcommand{\ovl}[1]{\overline{#1}}


%%Трансформация Фурье
\newcommand{\fourt}[1]{\mathcal{F}\left(#1\right)}
\newcommand{\ifourt}[1]{\mathcal{F}^{-1}\left(#1\right)}

%%Символ вектора
\newcommand{\vecm}[1]{\overrightarrow{#1\,}}

%%Пространстов матриц
\newcommand{\matsq}[1]{\operatorname{Mat}_{#1}}
\newcommand{\mat}[2]{\operatorname{Mat}_{#1, #2}}

%Оператор для действ и мнимых чисел
\DeclareMathOperator{\IM}{\operatorname{Im}}
\DeclareMathOperator{\RE}{\operatorname{Re}}
\DeclareMathOperator{\li}{\operatorname{li}}
\DeclareMathOperator{\GL}{\operatorname{GL}}
\DeclareMathOperator{\SL}{\operatorname{SL}}
\DeclareMathOperator{\Char}{\operatorname{char}}
\DeclareMathOperator\Arg{Arg}
\DeclareMathOperator\ord{ord}

%Оператор для образа
\DeclareMathOperator{\Ima}{Im}

%Делимость чисел
\newcommand{\modn}[3]{#1 \equiv #2 \; (\bmod \; #3)}
\newcommand{\nmodn}[3]{#1 \not\equiv #2 \; (\bmod \; #3)}

%%Взятие в скобки, модули и норму
\newcommand{\parfit}[1]{\left( #1 \right)}
\newcommand{\modfit}[1]{\left| #1 \right|}
\newcommand{\sqparfit}[1]{\left\{ #1 \right\}}
\newcommand{\normfit}[1]{\left\| #1 \right\|}

%%Функция для обозначения равномерной сходимости по множеству
\newcommand{\uconv}[1]{\overset{#1}{\rightrightarrows}}
\newcommand{\uconvm}[2]{\overset{#1}{\underset{#2}{\rightrightarrows}}}


%%Функция для обозначения нижнего и верхнего интегралов
\def\upint{\mathchoice%
	{\mkern13mu\overline{\vphantom{\intop}\mkern7mu}\mkern-20mu}%
	{\mkern7mu\overline{\vphantom{\intop}\mkern7mu}\mkern-14mu}%
	{\mkern7mu\overline{\vphantom{\intop}\mkern7mu}\mkern-14mu}%
	{\mkern7mu\overline{\vphantom{\intop}\mkern7mu}\mkern-14mu}%
	\int}
\def\lowint{\mkern3mu\underline{\vphantom{\intop}\mkern7mu}\mkern-10mu\int}

%%След матрицы
\DeclareMathOperator*{\tr}{tr}

\makeatletter
\renewcommand*\env@matrix[1][*\c@MaxMatrixCols c]{%
	\hskip -\arraycolsep
	\let\@ifnextchar\new@ifnextchar
	\array{#1}}
\makeatother


%% Переопределение функции хи, чтобы выглядела более приятно
\makeatletter
\@ifdefinable\@latex@chi{\let\@latex@chi\chi}
\renewcommand*\chi{{\@latex@chi\smash[t]{\mathstrut}}} % want only bottom half of \mathstrut
\makeatletter

\setcounter{MaxMatrixCols}{20}

\begin{document}
\lhead{Математический анализ - \RN{4}}
\chead{Шапошников С.В.}
\rhead{Лекция - 3}

\section*{Теорема Фубини}

\begin{theorem}(\textbf{Фубини})
	Пусть $\MI = \MI_x \times \MI_y$, $\MI_x \subset \MR^n, \, \MI_y \subset \MR^m$ - замкнутые бруски (в том числе $\MI$ тоже замкнутый брусок). Пусть $f$ интегрируема по Риману на $\MI$ и $\forall x \in \MI_x$ функция $y \mapsto f(x,y)$ интегрируема на бруске $\MI_y$, тогда функция: $x \mapsto \int_{\MI_y}f(x,y)dy$ интегрируема на $\MI_x$ и верно равенство:
	$$
		\iint\limits_{\MI} f(x,y)dxdy = \int\limits_{\MI_x}\Bigg(\int\limits_{\MI_y}f(x,y)dy\Bigg)dx
	$$
	Если $\forall y \in \MI_y$ функция $x \mapsto f(x,y)$ интегрируема на бруске $\MI_x$, тогда функция: $y \mapsto \int_{\MI_x}f(x,y)dx$ интегрируема на $\MI_y$ и верно равенство:
	$$
		\iint\limits_{\MI} f(x,y)dxdy = \int\limits_{\MI_y}\Bigg(\int\limits_{\MI_x}f(x,y)dx\Bigg)dy
	$$
\end{theorem}

В прошлый раз мы доказали эту теорему и как уже говорили, условие интегрируемости было крайне важным в этой теореме. 
\subsection*{Напоминание доказательства теоремы Фубини}
Мы проделали следующие шаги:
\begin{enumerate}[label=\arabic*)]
	\item Построили функции $h_n(x,y)$ - неубывающая, $g_n(x,y)$ - невозрастающая такие, что: 
	$$
		h_n(x,y) \leq f(x,y) \leq g_n(x,y), \, \ddint{\MI}{}h_n(x,y) - g_n(x,y)dxdy \to 0
	$$ 
	$$
		\ddint{\MI}{}h_n(x,y)dxdy \to \ddint{\MI}{}f(x,y)dxdy,  \quad \ddint{\MI}{}g_n(x,y)dxdy \to \ddint{\MI}{}f(x,y)dxdy
	$$
	\item Проинтегрировали эти ступенчатые функции и снова получили ступенчатые функции:
	$$
		H_n(x) = \ddint{\MI_y}{}h_n(x,y)dy, \quad G_n(x) = \ddint{\MI_y}{}g_n(x,y)dy,
	$$
	$$
		H_n(x) \leq \ddint{\MI_y}{}f(x,y)dy \leq G_n(x)
	$$
	\item Поняли, что $H_n(x)$ - неубывающая, $G_n(x)$ - невозрастающая. А поскольку для индикатора бруска, а следовательно и для ступенчатой функции теорема выполнена, то верно:
	$$
		\ddint{\MI_x}{}H_n(x)dx = \iint\limits_{\MI}{}h_n(x,y)dxdy \to \iint\limits_{\MI}{}f(x,y)dxdy \gets \iint\limits_{\MI}{}g_n(x,y)dxdy = \ddint{\MI_x}{}G_n(x)dx 
	$$
	Далее критерий интегрируемости даёт нам интегрируемость $\int_{\MI_y}{}f(x,y)dy$ как функции $x$ по $x$ и требуемое равенство;
\end{enumerate}

\subsection*{Обобщение теоремы Фубини для интеграла Римана}

Вопрос, а что делать если интеграл не существует. Пусть теперь нет условия, что $\exists \, \int_{\MI_y}f(x,y)dy$. Введём следующую функцию:
$$
	F(x) = 
	\begin{cases}
		\int_{\MI_y}f(x,y)dy, & \text{интеграл существует}\\
		A, & \text{интеграл не существует}
	\end{cases}
$$
Что нам необходимо добавить? Мы хотим, чтобы также выполнялось:
$$
	H_n(x) \leq F(x) \leq G_n(x)
$$
Тогда мы сразу же получаем, что:
$$
	\iint\limits_{\MI}{}f(x,y)dxdy = \ddint{\MI_x}{}F(x)dx
$$
Нам необходимо доопределить $F(x)$ так, чтобы всё время выполнялись неравенства выше, для всех $n$. Подойдет следующее число $A$:
$$
	A \in \left[\sup\limits_{n}H_n(x), \, \inf\limits_{n}G_n(x)\right]
$$
где начало и конец отрезка понимаем как пределы. Пределы $H_n(x)$ и $G_n(x)$ всегда существуют, поскольку это ограниченные, монотонные последовательности. В том случае когда интеграл существует, то точная верхняя грань $H_n(x)$ и точная нижняя грань $G_n(x)$ совпадут:
$$
	\lim \ddint{\MI_y}{}(h_n(x,y) - g_n(x,y))dy \xrightarrow[n\to \infty]{} 0
$$
Но это означает, что интеграл по $f$ существует (по критерию интегрируемости). Следовательно, можно было бы заменить последним условием про $A$ всю функцию.
\begin{rem}
	Теорема Фубини верна, если отменить условие интегрируемости $y \mapsto f(x,y)$ (аналогично для $x \mapsto f(x,y)$) и заменить $\int_{\MI_y}f(x,y)dy$ на $F(x)$ (или заменить $\int_{\MI_x}f(x,y)dx$ на $F(y)$). Тем не менее, плата за такую замену - большая, вместо повторного интеграла мы получаем:
	$$
		\iint\limits_{\MI}{}f(x,y)dxdy = \ddint{\MI_x}{}F(x)dx
	$$
	И на каждом сечении нам потребуется проверять интегрируемость.
\end{rem}

Возникает вопрос, а как много таких точек, в которых потребуется доопределение? Для начала вспомним определение из второго семестра.
\begin{defn}
	Множество $E \subset \MR$ называется \uwave{множеством меры ноль по Лебегу}, если: $\forall \VE > 0, \, \exists$ не более чем счетный набор интервалов $\{\MI_n\}$ таких, что:
	\begin{enumerate}[label={(\arabic*)}]
		\item Множество $E$ покрыто этими интервалами: $E \subset \bigcup_n \MI_n$;
		\item Сумма длин этих интервалов меньше $\VE$: $\sum_{n} |\MI_n| < \VE$;
	\end{enumerate}
\end{defn}
\begin{defn}
	Если некоторое свойство имеет место для всех точек, кроме множества меры ноль, то говорят, что это \uwave{свойство выполняется почти всюду}.
\end{defn}
\begin{rem}
	Заметим, что далее мы это ещё обсудим, но в общем случае, вместо отрезков будут бруски.
\end{rem}

\begin{prop}
	Множество $x \colon \nexists \, \int_{\MI_y}f(x,y)dy$ имеет меру нуль по Лебегу.
\end{prop}

\begin{proof}
	Рассмотрим следующие функции:
	\begin{enumerate}[label=\arabic*)]
		\item $F^+$, доопределив её в ``плохих'' точках значениями $\inf\limits_n G_n(x)$;
		\item $F^-$, доопределив её в ``плохих'' точках значениями $\sup\limits_n H_n(x)$;
	\end{enumerate}
	Тогда очевидно, что: $F^{-}(x) \leq F^{+}(x)$ и по замечанию к теореме Фубини выше будет верно:
	$$
		\iint\limits_{\MI}f(x,y)dxdy = \ddint{\MI_x}{}F^+(x)dx =  \ddint{\MI_x}{}F^-(x)dx \Rightarrow
	$$
	$$
		\Rightarrow \ddint{\MI_x}{}(\underbrace{F^+(x) - F^-(x)}_{\geq 0})dx = 0 \Rightarrow F^+(x) - F^-(x) = 0 \text{ почти всюду}
	$$
	где последнее верно по аналогии со следствием $4$ лекции $25$ семестра $2$.
\end{proof}
\begin{rem}
	Далее мы приведём более строгое доказательство для брусков вместо отрезков.
\end{rem}

\textbf{\uline{Вывод}}: Теорема Фубини хороша в первоначальной формулировке, но если мы всё же не хотим требовать, чтобы на сечениях существовал интеграл, то смысл будет такой: если $f$ интегрируема на $\MI$, то существует интегрируемая на $\MI_x$ функция, которая почти всюду совпадает с интегралом $\int_{\MI_y}f(x,y)dy$ (и который тем самым объявляется существующим) и верно равенство:
$$
	\iint\limits_{\MI}f(x,y)dxdy = \ddint{\MI_x}{}\bigg(\ddint{\MI_y}{}f(x,y)dy\bigg)dx
$$ 

\begin{prop}
	Пусть $\MI = \MI_x \times \MI_y$ и $f \in C(\MI)$, тогда $\exists \, \int_{\MI_x}f(x,y)dx, \, \int_{\MI_y}f(x,y)dy$ и функция $x \mapsto \int_{\MI_y}f(x,y)dy$ непрерывна на $\MI_x$, функция $y \mapsto \int_{\MI_x}f(x,y)dx$ непрерывна на $\MI_y$.
\end{prop}
\begin{proof}
	Функция непрерывна на $\MI \Rightarrow$ она непрерывна по каждой переменной $\Rightarrow$ все интегралы существуют. Рассмотрим функцию: $x \mapsto \int_{\MI_y}f(x,y)dy$, пусть $x_n \to x_0$, тогда из-за равномерной непрерывности $f$ на $\MI$ (замкнутый брус это компакт $\Rightarrow f$ - равномерно непрерывна) мы получаем:
	$$
		f(x_n,y)\uconvm{\MI_y}{n \to \infty} f(x_0,y)
	$$
	Тогда по теореме о равномерном пределе под интегралом, мы получаем:
	$$
		\ddint{\MI_y}{}f(x_n,y)dy \xrightarrow[n \to \infty]{} \ddint{\MI_y}{}f(x_0,y)dy
	$$
	Аналогично для функции: $y \mapsto \int_{\MI_x}f(x,y)dx$.
\end{proof}

\begin{corollary}
	Пусть $\MI = [a_1,b_1]\times \dotsc \times [a_n,b_n]$, $f \in C(\MI)$. Тогда для любой перестановки $(i_1, i_2,\dotsc,i_n)$ чисел $(1,2,\dotsc,n)$ верно равенство:
	$$
		\ddint{\MI}{}f(x)dx = \ddint{a_{i_1}}{b_{i_1}}\Bigg(\ddint{a_{i_2}}{b_{i_2}}\Bigg( \dotsc \Bigg(   \ddint{a_{i_n}}{b_{i_n}}f(x)dx_{i_n}\Bigg)  \dotsc  \Bigg)   dx_{i_2}\Bigg) dx_{i_1}
	$$
\end{corollary}
\begin{rem}
	Без непрерывности придется каждый раз доопределять функцию по аналогии с доопределением теоремы Фубини.
\end{rem}
\begin{proof}
	Рассмотрим тождественную перестановку: $i_1 = 1, i_2 = 2, \dotsc, i_n = n$. По индукции докажем, что:
	$$
		\ddint{\MI}{}f(x)dx = \ddint{a_1}{b_1}\Bigg(\ddint{a_2}{b_2}\Bigg( \dotsc \Bigg(   \ddint{a_n}{b_n}f(x)dx_n\Bigg)  \dotsc  \Bigg)   dx_2\Bigg) dx_1
	$$
	\uline{База индукции}: Для $n = 1$ - очевидно, для $n = 2$ - теорема Фубини:
	$$
		\ddint{a_1}{b_1}f(x)dx_1 = \ddint{a_1}{b_1}f(x)dx_1, \, \ddint{\MI}{}f(x)dx_1dx_2 = \ddint{a_1}{b_1}\Bigg( \ddint{a_2}{b_2}f(x) dx_2\Bigg)dx_1
	$$
	\uline{Шаг индукции}: Пусть верно для $\leq n-1$. Пусть: $\MI = [a_1,b_1]\times \MI_{n-1}, \, \MI_{n-1} = [a_2,b_2]\times \dotsc \times [a_n,b_n]$, тогда из непрерывности функции $f$ на $\MI$ и по теореме Фубини будет верно:
	$$
		\ddint{\MI}{}f(x)dx = \ddint{a_1}{b_1}\Bigg( \ddint{\MI_{n-1}}{}f(x_1,x_2,\dotsc,x_n)dx_2dx_3\dotsc dx_n \Bigg)dx_1
	$$
	Расписываем внутренний интеграл по индукции и получаем требуемое. Для произвольной перестановки необходимо уметь делать транспозицию. Пусть утверждение верно для какого-либо порядка:
	$$
		\ddint{\MI}{}f(x)dx = \ddint{a_{i_1}}{b_{i_1}}\Bigg(\ddint{a_{i_2}}{b_{i_2}}\Bigg( \dotsc \Bigg(   \ddint{a_{i_n}}{b_{i_n}}f(x)dx_{i_n}\Bigg)  \dotsc  \Bigg)   dx_{i_2}\Bigg) dx_{i_1}
	$$
	Достаточно уметь делать транспозицию соседних интегралов: рассмотрим $i_k$ и $i_{k+1}$:
	$$
		\ddint{a_{i_k}}{b_{i_k}}\Bigg(  \ddint{a_{i_{k+1}}}{b_{i_{k+1}}} F(x)dx_{i_{k+1}}\Bigg)dx_{i_k} = \ddint{\MI_{i_k} \times \MI_{i_{k+1}}}{} F(x)dx_{i_k}dx_{i_{k+1}} = \ddint{a_{i_{k+1}}}{b_{i_{k+1}}} \Bigg( \ddint{a_{i_k}}{b_{i_k}}  F(x)dx_{i_k}\Bigg)dx_{i_{k+1}}
	$$
	где $F$ это интеграл от $f$ по некоторому бруску $\Rightarrow F$ это непрерывная функция $\Rightarrow$ интегрируемая на прямоугольнике: $\MI_{i_k} \times \MI_{i_{k+1}} = [a_{i_k},b_{i_k}]\times[a_{i_{k+1}},b_{i_{k+1}}] \Rightarrow$ можем поменять порядок по теореме Фубини.
\end{proof}
\newpage
\subsection*{Формула интегрирования по частям}

\begin{theorem}(\textbf{Формула интегрирования по частям})
	Пусть $f, g \in C^1(\MI)$ и $f = 0$ на границе $\MI$ (на гранях), тогда для всякого $k$ верно равенство:
	$$
		\ddint{\MI}{}f(x){\cdot}\dfrac{\partial g}{\partial x_k}(x)dx = - \ddint{\MI}{}g(x){\cdot}\dfrac{\partial f}{\partial x_k}(x)dx
	$$
\end{theorem}
\begin{rem}
	Заметим, что здесь нет внеинтегрального члена, поскольку $f = 0$. Когда $f \neq 0$, то лучше это обсуждать в теме поверхностного интегрирования.
\end{rem}
\begin{proof}
	Пусть $\MI = [a_1,b_1]\times \dotsc \times [a_n,b_n]$, тогда:
	$$
		\ddint{\MI}{}f(x){\cdot}\dfrac{\partial g}{\partial x_k}(x)dx = \underbrace{\ddint{a_1}{b_1} \dotsc}_{\text{все кроме } k} \Bigg( \ddint{a_k}{b_k}f(x)\dfrac{\partial g}{\partial x_k}dx_k \Bigg)\dotsc dx_1
	$$
	Поскольку все $x_i$ фиксированы, кроме $x_k$, то применим формулу интегрирования по частям к внутреннему интегралу, с учетом того, что $f(x) = 0$ на гранях $\MI$:
	$$
		\ddint{a_k}{b_k}f(x){\cdot}\dfrac{\partial g}{\partial x_k}(x)dx_k =f(x){\cdot}g(x)\Big\vert_{x_k = a_k}^{b_k} - \ddint{a_k}{b_k}\dfrac{\partial f}{\partial x_k}(x){\cdot}g(x)dx_k = - \ddint{a_k}{b_k}\dfrac{\partial f}{\partial x_k}(x){\cdot}g(x)dx_k \Rightarrow
	$$
	$$
		\Rightarrow \ddint{a_1}{b_1} \dotsc \Bigg( \ddint{a_k}{b_k}f(x)\dfrac{\partial g}{\partial x_k}dx_k \Bigg)\dotsc dx_1 = -\ddint{a_1}{b_1} \dotsc \Bigg( \ddint{a_k}{b_k}\dfrac{\partial f}{\partial x_k}(x){\cdot}g(x)dx_k \Bigg)\dotsc dx_1 = - \ddint{\MI}{}g(x){\cdot}\dfrac{\partial f}{\partial x_k}(x)dx
	$$
	где в последнем равенстве мы опять воспользовались теоремой Фубини в силу непрерывности $f$ на $\MI$.
\end{proof}

\begin{corollary}
	Пусть $f,g \in C^m(\MI)$, $f,f^{(1)}, \dotsc, f^{(m-1)} = 0$ на границе $\MI$, тогда $\forall (i_1,\dotsc,i_m)$:
	$$
		\ddint{\MI}{}f(x){\cdot}\dfrac{\partial^m g}{\partial x_{i_1}\dotsc \partial x_{i_m}}(x)dx = (-1)^m{\cdot}\ddint{\MI}{}\dfrac{\partial^m f(x)}{\partial x_{i_1}\dotsc \partial x_{i_m}}{\cdot}g(x)dx
	$$
\end{corollary}
\begin{proof}
	Доказывается по индукции с применением формулы интегрирования по частям.
\end{proof}

\begin{rem}
	Отметим, что писать так: $\int_{\MI}f(\vecm{x})d\vecm{x}$ - некорректно, правда раньше так писали. Сейчас запись преимущественно имеет вид: $\int_{\MI}f(x)dx$ или $\int_{\MI}f$, но последняя запись очень плоха для геометрии, так как надо писать дифференциальную форму по которой идёт интегрирование. 
	
	Также заметим, что сейчас мы под $dx$ понимаем: $dx_1{\cdot}\dotsc{\cdot}dx_n$, но далее мы будем воспринимать $dx$ как слитный символ, где $dx_1, \dotsc ,dx_n$ не разделены. Они распадутся в отдельные интегралы по $x_1,\dotsc,x_n$, когда будет применена теорема Фубини. Таким образом, можно $dx$ воспринимать как элемент объема. Символ $dx$ не будет произведением: $dx_1,\dotsc, dx_n$, поскольку тела не обязана иметь вид бруска, например, у шара это уже никакое не произведение -  оно произведение в смысле теоремы Фубини.
\end{rem}
\newpage

\section*{Критерий Лебега}

Ранее мы вывели критерий интегрируемости и из него вывели теорему Фубини и её следствия. Критерий подразумевает, что мы легко умеем строить ступенчатые функции и нам очевидно, стремятся интегралы от них к одному и тому же или нет. Делать это не очень удобно и более того совершать проверки для произвольных функций даже на $3$-хмерных брусках достаточно сложно. Таким образом, хотелось бы какой-нибудь простой критерий. Для одномерных случаев таким был критерий Лебега.

\subsection*{Множество меры нуль по Лебегу}
\begin{defn}
	Множество $E \subset \MR^n$ называется \uwave{множеством меры нуль по Лебегу}, если: $\forall \VE > 0, \, \exists \, $ не более чем счетный набор замкнутых брусков $\{\MI_k\}$ такой, что:
	\begin{enumerate}[label=\arabic*)]
		\item $E \subset \bigcup_k \MI_k$;
		\item $\sum_k |\MI_k| < \VE$;
	\end{enumerate}
\end{defn}
\textbf{Примеры}: 
\begin{enumerate}[label=\arabic*)]
	\item Точка - множество меры нуль по Лебегу;
	\item Конечный набор точек - множество меры нуль по Лебегу;
\end{enumerate}

\begin{prop}
	В определении множества меры нуль по Лебегу замкнутые брусы: $[a_1,b_1]\times \dotsc \times [a_n,b_n]$ можно заменить на открытые: $(a_1,b_1)\times \dotsc \times (a_n,b_n)$.
\end{prop}
\begin{proof}\hfill\\
	$(\Leftarrow)$ Если $E$ покрыли открытыми $\MJ_k \colon \sum_k |\MJ_k| < \VE$, то $E$ покрыто замкнутыми брусками: $\ovl{\MJ}_k,\, |\MJ_k| = |\ovl{\MJ}_k|$:
	$$
		E \subset \bigcup\limits_k \MJ_k \subseteq \bigcup\limits_k \ovl{\MJ}_k, \, \ddsum{k}{}|\MJ_k| = \ddsum{k}{}|\ovl{\MJ}_k| < \VE
	$$
	$(\Rightarrow)$ Если $E$ покрыли замкнутыми $\MI_k \colon \sum_k |\MI_k| < \VE$, то:
	$$
		\MI_k = [a_1^k, b_1^k] \times \dotsc \times [a_n^k, b_n^k] \subset \MJ_k = (\alpha_1^k,\beta_1^k) \times \dotsc \times (\alpha_n^k,\beta_n^k), \quad \beta_i^k - \alpha_i^k = 3 (b_i^k - a_i^k) \Rightarrow
	$$
	$$
		\Rightarrow |\MJ_k| = 3^n{\cdot}|\MI_k| \Rightarrow \ddsum{k}{}|\MJ_k| < 3^n\VE
	$$
	Так как $\VE$ - произвольное, то умеем покрывать открытыми $\forall \VE > 0$.
\end{proof}

\begin{corollary}
	Пусть $\MI = [a_1,b_1]\times \dotsc \times[a_n,b_n]$, причём верно: $\forall k, \, b_k - a_k > 0$. Тогда $\MI$ не является множеством меры нуль.
\end{corollary}
\begin{proof}
	Предположим противное, пусть: 
	$$
		0 < \VE < |\MI|, \, \MI \subset \bigcup_k \MJ_k, \, \MJ_k \text{ - открытые}, \, \sum_k |\MJ_k| < \VE
	$$ 
	Поскольку $\MI$ это компакт, то можно считать, что:
	$$
		\MI \subset \bigcup\limits_{k = 1}^{N}\MJ_k \Rightarrow |\MI | \leq \ddsum{k = 1}{N}|\MJ_k| < \VE
	$$
	Получили противоречие.
\end{proof}

\begin{prop}
	Пусть $\MI$ - замкнутый брус в $\MR^n$ и $f\in C(\MI)$, тогда $\Gamma_f = \{(x,y) \mid y = f(x), \, x \in \MI\}$ - график функции $f$ является множеством меры нуль по Лебегу в $\MR^{n+1}$.
\end{prop}
\begin{proof}
	Поскольку $f \in C(\MI)$, то $f$ - равномерно непрерывна на $\MI$, тогда разбиваем $\MI$ на бруски $\{\MI_k\}$ так, чтобы:
	$$
		\forall \VE > 0, \, \exists \, \delta > 0 \colon \diam(\MI_k) < \delta \Rightarrow \forall x, \wte{x} \in \MI_k, \, |f(x) - f(\wte{x})| < \VE
	$$
	Последнее верно, поскольку мы берём разбиение, диаметр которого меньше $\delta$. Следовательно:
	$$
		\forall k, \, \exists \, [\alpha_k,\beta_k] \colon \beta_k - \alpha_k = 2\VE, \, \{(x,f(x)) \mid x \in \MI_k\} \subset \MI_k \times [\alpha_k, \beta_k] = \MJ_k \Rightarrow |\MJ_k| = |\MI_k|{\cdot}2\VE \Rightarrow
	$$
	$$
		\Rightarrow \Gamma_f \subset \bigcup\limits_k \MJ_k, \, \ddsum{k}{}|\MJ_k| = 2\VE{\cdot}\ddsum{k}{}|\MI_k| = 2\VE{\cdot}|\MI|
	$$
	В силу произвольности $\VE$ мы получаем требуемое.
\end{proof}
\begin{rem}
	Полезно иметь в виду, что графики всех разумных функций являются множеством меры ноль и это в определенном смысле из продвинутой теоремы Фубини. Но мы пока не можем это доказать нашими средствами.
\end{rem}
\begin{rem}
	Тем самым любые плоскости, гиперплоскости это множество меры нуль в пространстве большей размерности. Или всё что мы нарисуем меньшей размерности (в условиях теоремы), чем размерность пространства, будет множеством меры нуль.
\end{rem}

\subsection*{Свойства множеств меры нуль}
\begin{prop}(\textbf{Свойства множеств меры нуль})\hfill
	\begin{enumerate}[label=\arabic*)]
		\item Если $D$ - множество меры нуль по Лебегу и $E \subset D$, то $E$ - множество меры нуль;
		\item Если $\{E_n\}$ - не более чем счётный набор множеств меры нуль, то $\bigcup_n E_n$ - множество меры нуль;
	\end{enumerate}
\end{prop}
\begin{proof}\hfill
	\begin{enumerate}[label=\arabic*)]
		\item Очевидно, поскольку накрыли $D \Rightarrow$ накрыли и $E \subset D$ так, что сумма объемов $< \VE$;
		\item Возьмем произвольный $\VE > 0$, накроем $E_n$ брусками: $E_m \subset \bigcup_k \MI_k^m \colon \sum_k |\MI_k^m| < \tfrac{\VE}{2^m}$. Тогда:
		$$
			\bigcup_m E_m \subset \bigcup_{k,m} \MI_k^m, \quad \ddsum{k,m}{}|\MI_k^m| < \VE{\cdot}\ddsum{m}{}\dfrac{1}{2^m} < \VE 
		$$
	\end{enumerate}	
\end{proof}

\begin{defn}
	Если некоторое свойства выполняется для всех точек $x$, кроме точек множества меры нуль, то говорят, что это свойство выполняется \uwave{почти всюду}.
\end{defn}

\begin{prop}
	Если $f$ интегрируема на $\MI$ и $f = 0$ почти всюду, тогда верно:
	$$
		\ddint{\MI}{}f(x)dx = 0
	$$
\end{prop}
\begin{proof}
	Пусть $\{\MI_i\}$ это разбиение $\MI$. По условию, поскольку $f$ интегрируема, то:
	$$
		\sigma(f,\MTB,\xi) = \ddsum{i}{}f(\xi_i){\cdot}|\MI_i| \xrightarrow[\lambda(\MTB) \to 0]{} \ddint{\MI}{}f(x)dx
	$$
	Множество $\MI_i$ - не множество меры нуль $\Rightarrow \exists \, \xi \in \MI_i \colon f(\xi) = 0$, но поскольку Риманова сумма не зависит от выбора $\xi_i$, то мы получим:
	$$
		\sigma(f,\MTB,\xi) = \ddsum{i}{}0{\cdot}|\MI_i| = 0 \xrightarrow[\lambda(\MTB) \to 0]{} 0 \Rightarrow \ddint{\MI}{}f(x)dx = 0
	$$
\end{proof}
\subsection*{Критерий Лебега}
\begin{theorem}(\textbf{Критерий Лебега})
	$f$ интегрируема по Риману на $\MI \Leftrightarrow f$ - ограничена и $f$ - непрерывна почти всюду на $\MI$.
\end{theorem}

\end{document}