\documentclass[12pt]{article}
\usepackage[left=1cm, right=1cm, top=2cm,bottom=1.5cm]{geometry} 

\usepackage[parfill]{parskip}
\usepackage[utf8]{inputenc}
\usepackage[T2A]{fontenc}
\usepackage[russian]{babel}
\usepackage{enumitem}
\usepackage[normalem]{ulem}
\usepackage{amsfonts, amsmath, amsthm, amssymb, mathtools,xcolor,accents}
\usepackage{blkarray}

\usepackage{tabularx}
\usepackage{hhline}

\usepackage{accents}
\usepackage{fancyhdr}
\pagestyle{fancy}
\renewcommand{\headrulewidth}{1.5pt}
\renewcommand{\footrulewidth}{1pt}

\usepackage{graphicx}
\usepackage[figurename=Рис.]{caption}
\usepackage{subcaption}
\usepackage{float}

%%Наименование папки откуда забирать изображения
\graphicspath{ {./images/} }

%%Изменение формата для ввода доказательства
\renewcommand{\proofname}{$\square$  \nopunct}
\renewcommand\qedsymbol{$\blacksquare$}

%%Изменение отступа на таблицах
\addto\captionsrussian{%
	\renewcommand{\proofname}{$\square$ \nopunct}%
}
%% Римские цифры
\newcommand{\RN}[1]{%
	\textup{\uppercase\expandafter{\romannumeral#1}}%
}

%% Для удобства записи
\newcommand{\MR}{\mathbb{R}}
\newcommand{\MC}{\mathbb{C}}
\newcommand{\MQ}{\mathbb{Q}}
\newcommand{\MN}{\mathbb{N}}
\newcommand{\MZ}{\mathbb{Z}}
\newcommand{\MTB}{\mathbb{T}}
\newcommand{\MTI}{\mathbb{I}}
\newcommand{\MI}{\mathrm{I}}
\newcommand{\MCI}{\mathcal{I}}
\newcommand{\MJ}{\mathrm{J}}
\newcommand{\MH}{\mathrm{H}}
\newcommand{\MT}{\mathrm{T}}
\newcommand{\MU}{\mathcal{U}}
\newcommand{\MV}{\mathcal{V}}
\newcommand{\MA}{\mathcal{A}}
\newcommand{\MB}{\mathcal{B}}
\newcommand{\MF}{\mathcal{F}}
\newcommand{\ME}{\mathcal{E}}
\newcommand{\MW}{\mathcal{W}}
\newcommand{\ML}{\mathcal{L}}
\newcommand{\MP}{\mathcal{P}}
\newcommand{\VN}{\varnothing}
\newcommand{\VE}{\varepsilon}
\newcommand{\dx}{\, dx}
\newcommand{\dy}{\, dy}
\newcommand{\dz}{\, dz}
\newcommand{\dd}{\, d}


\theoremstyle{definition}
\newtheorem{defn}{Опр:}
\newtheorem{rem}{Rm:}
\newtheorem{prop}{Утв.}
\newtheorem{exrc}{Упр.}
\newtheorem{problem}{Задача}
\newtheorem{lemma}{Лемма}
\newtheorem{theorem}{Теорема}
\newtheorem{corollary}{Следствие}

\newenvironment{cusdefn}[1]
{\renewcommand\thedefn{#1}\defn}
{\enddefn}

\DeclareRobustCommand{\divby}{%
	\mathrel{\text{\vbox{\baselineskip.65ex\lineskiplimit0pt\hbox{.}\hbox{.}\hbox{.}}}}%
}
\DeclareRobustCommand{\ndivby}{\mkern-1mu\not\mathrel{\mkern4.5mu\divby}\mkern1mu}


%Короткий минус
\DeclareMathSymbol{\SMN}{\mathbin}{AMSa}{"39}
%Длинная шапка
\newcommand{\overbar}[1]{\mkern 1.5mu\overline{\mkern-1.5mu#1\mkern-1.5mu}\mkern 1.5mu}
%Функция знака
\DeclareMathOperator{\sgn}{sgn}

%Функция ранга
\DeclareMathOperator{\rk}{\text{rk}}
\DeclareMathOperator{\diam}{\text{diam}}


%Обозначение константы
\DeclareMathOperator{\const}{\text{const}}

\DeclareMathOperator{\codim}{\text{codim}}

\DeclareMathOperator*{\dsum}{\displaystyle\sum}
\newcommand{\ddsum}[2]{\displaystyle\sum\limits_{#1}^{#2}}
\newcommand{\ddssum}[2]{\displaystyle\smashoperator{\sum\limits_{#1}^{#2}}}
\newcommand{\ddlsum}[2]{\displaystyle\smashoperator[l]{\sum\limits_{#1}^{#2}}}
\newcommand{\ddrsum}[2]{\displaystyle\smashoperator[r]{\sum\limits_{#1}^{#2}}}

%Интеграл в большом формате
\DeclareMathOperator{\dint}{\displaystyle\int}
\newcommand{\ddint}[2]{\displaystyle\int\limits_{#1}^{#2}}
\newcommand{\ssum}[1]{\displaystyle \sum\limits_{n=1}^{\infty}{#1}_n}

\newcommand{\smallerrel}[1]{\mathrel{\mathpalette\smallerrelaux{#1}}}
\newcommand{\smallerrelaux}[2]{\raisebox{.1ex}{\scalebox{.75}{$#1#2$}}}

\newcommand{\smallin}{\smallerrel{\in}}
\newcommand{\smallnotin}{\smallerrel{\notin}}

\newcommand*{\medcap}{\mathbin{\scalebox{1.25}{\ensuremath{\cap}}}}%
\newcommand*{\medcup}{\mathbin{\scalebox{1.25}{\ensuremath{\cup}}}}%

\makeatletter
\newcommand{\vast}{\bBigg@{3.5}}
\newcommand{\Vast}{\bBigg@{5}}
\makeatother

%Промежуточное значение для sup\inf, поскольку они имеют разную высоту
\newcommand{\newsup}{\mathop{\smash{\mathrm{sup}}}}
\newcommand{\newinf}{\mathop{\mathrm{inf}\vphantom{\mathrm{sup}}}}

%Скалярное произведение
\newcommand{\inner}[2]{\left\langle #1, #2 \right\rangle }
\newcommand{\linsp}[1]{\left\langle #1 \right\rangle }
\newcommand{\linmer}[2]{\left\langle #1 \vert #2\right\rangle }

%Подпись символов снизу
\newcommand{\ubar}[1]{\underaccent{\bar}{#1}}

%%Шапка для букв сверху
\newcommand{\wte}[1]{\widetilde{#1}}
\newcommand{\wht}[1]{\widehat{#1}}
\newcommand{\ovl}[1]{\overline{#1}}


%%Трансформация Фурье
\newcommand{\fourt}[1]{\mathcal{F}\left(#1\right)}
\newcommand{\ifourt}[1]{\mathcal{F}^{-1}\left(#1\right)}

%%Символ вектора
\newcommand{\vecm}[1]{\overrightarrow{#1\,}}

%%Пространстов матриц
\newcommand{\matsq}[1]{\operatorname{Mat}_{#1}}
\newcommand{\mat}[2]{\operatorname{Mat}_{#1, #2}}

%Оператор для действ и мнимых чисел
\DeclareMathOperator{\IM}{\operatorname{Im}}
\DeclareMathOperator{\RE}{\operatorname{Re}}
\DeclareMathOperator{\li}{\operatorname{li}}
\DeclareMathOperator{\GL}{\operatorname{GL}}
\DeclareMathOperator{\SL}{\operatorname{SL}}
\DeclareMathOperator{\Char}{\operatorname{char}}
\DeclareMathOperator\Arg{Arg}
\DeclareMathOperator\ord{ord}

%Оператор для образа
\DeclareMathOperator{\Ima}{Im}

%Делимость чисел
\newcommand{\modn}[3]{#1 \equiv #2 \; (\bmod \; #3)}
\newcommand{\nmodn}[3]{#1 \not\equiv #2 \; (\bmod \; #3)}

%%Взятие в скобки, модули и норму
\newcommand{\parfit}[1]{\left( #1 \right)}
\newcommand{\modfit}[1]{\left| #1 \right|}
\newcommand{\sqparfit}[1]{\left\{ #1 \right\}}
\newcommand{\normfit}[1]{\left\| #1 \right\|}

%%Функция для обозначения равномерной сходимости по множеству
\newcommand{\uconv}[1]{\overset{#1}{\rightrightarrows}}
\newcommand{\uconvm}[2]{\overset{#1}{\underset{#2}{\rightrightarrows}}}

%% Функция для добавления круга сверху множества
\newcommand{\Circ}[1]{\accentset{\circ}{#1}}

%%Функция для обозначения нижнего и верхнего интегралов
\def\upint{\mathchoice%
	{\mkern13mu\overline{\vphantom{\intop}\mkern7mu}\mkern-20mu}%
	{\mkern7mu\overline{\vphantom{\intop}\mkern7mu}\mkern-14mu}%
	{\mkern7mu\overline{\vphantom{\intop}\mkern7mu}\mkern-14mu}%
	{\mkern7mu\overline{\vphantom{\intop}\mkern7mu}\mkern-14mu}%
	\int}
\def\lowint{\mkern3mu\underline{\vphantom{\intop}\mkern7mu}\mkern-10mu\int}

%%След матрицы
\DeclareMathOperator*{\tr}{tr}

\makeatletter
\renewcommand*\env@matrix[1][*\c@MaxMatrixCols c]{%
	\hskip -\arraycolsep
	\let\@ifnextchar\new@ifnextchar
	\array{#1}}
\makeatother


%% Переопределение функции хи, чтобы выглядела более приятно
\makeatletter
\@ifdefinable\@latex@chi{\let\@latex@chi\chi}
\renewcommand*\chi{{\@latex@chi\smash[t]{\mathstrut}}} % want only bottom half of \mathstrut
\makeatletter

\setcounter{MaxMatrixCols}{20}

\begin{document}
\lhead{Математический анализ - \RN{4}}
\chead{Шапошников С.В.}
\rhead{Лекция - 10}

\section*{Меры. Внешние меры}
\textbf{\uline{Цель}}: Мера Лебега и мера Хаусдорфа. Меры Лебега хватает, чтобы говорить об объемах без тонкостей связанных с интегралом Римана и мерой Жордана. Мера Хаусдорфа это правильный способ говорить о поверхностных площадях, комерных объемах и так далее. 

Таким образом, если хотим проинтегрировать в $\MR^n$ по какому-нибудь необычному множеству $\Rightarrow$ мера Лебега, если хотим проинтегрировать в $\MR^n$ или в метрическом пространстве по какому-то объекту и хотим, чтобы мера учитывала его геометрию (например, выражала площадь поверхности или длину кривой), то надо использовать меру Хаусдорфа. 

\begin{defn}
	Пусть $X \neq \VN$, набор подмножеств $\mathcal{A}$ множества $X$ называется \uwave{алгеброй}, если:
	\begin{enumerate}[label=\arabic*)]
		\item $\VN, X \in \mathcal{A}$;
		\item $A,B \in \mathcal{A} \Rightarrow A \cap B, \, A \setminus B, \, A \cup B \in \mathcal{A}$;
	\end{enumerate} 
	Если дополнительно верно, что:
	\begin{enumerate}[label=\arabic*)]
		\setcounter{enumi}{2}
		\item $\forall n \in \MN, \, A_n \in \mathcal{A} \Rightarrow \bigcup\limits_n A_n, \, \bigcap\limits_n A_n \in \mathcal{A}$;
	\end{enumerate} 
	
	то $\mathcal{A}$ называется \uwave{$\sigma$-алгеброй}.
\end{defn}

\textbf{Примеры алгебр}:
\begin{enumerate}[label=\arabic*)]
	\item $\{\VN,X\}$ - $\sigma$-алгебра;
	\item $2^X$ - $\sigma$-алгебра;
	\item Возьмем $B \in X$ и набор $\{\VN,B, X \setminus B, X \}$ - $\sigma$-алгебра; 
	\item $\{\text{конечные объединения промежутков из } [0,1]\}$ - алгебра, но не $\sigma$-алгебра;
	\begin{proof}
		Пересечение промежутков - это промежуток, объединение промежутков это объединение промежутков. Дополнение к промежутку это либо промежуток, либо объединение двух промежутков. Дополнение к конечному объединению $\Rightarrow$ пересечение дополнений $\Rightarrow$ пересечение объектов из данного набора. Рассмотрим рациональные числа:
		$$
		\MQ = \bigcup\limits_n\{r_n\} \overset{?}{=} \bigcup\limits_{1}^N\MI_k
		$$
		Это будет не верно, поскольку если $\MI_k$ не является точкой, то в $\MI_k$ есть иррациональное число $\Rightarrow$ должны быть точками $\Rightarrow$ получается конечный набор точек, а $\MQ$ - счётное множество $\Rightarrow$ не являтеся $\sigma$-алгеброй.
	\end{proof} 
	\item $\{\VN,\MN, \text{конечные множества и дополнения к конечным множествам в }\MN\}$ - алгебра, но не $\sigma$-алгебра; 
	\begin{proof}
		Это можно показать взяв множество всех чётных чисел - оно есть счётное объединение множеств из одного элмента (в отдельности каждого чётного числа), но при этом оно не конечное и не является дополнением к конечному.
	\end{proof}
\end{enumerate}
\begin{rem}
	В теории вероятности к $\sigma$-алгебре относятся события. Поскольку интересуют обычно вопросы асимптотические, что происходит, когда количество событий - очень большое (бесконечное обычно) $\Rightarrow$ надо уметь что-то делать не только с конечным набором, но и с счётным $\Rightarrow$ рассматриваются $\sigma$-алгебры в качестве множества событий.
\end{rem}
\begin{rem}
	Аналогично, с точки зрения вычисления объемов, площадей и длин, $\sigma$-алгебра также естественный объект потому, что сложные объекты получаются из простых $\Rightarrow$ минимальный набор действий в алгебре.
\end{rem}

\begin{defn}
	Пусть $S$ - какой-либо непустой набор подмножеств $X$, тогда: 
	$$
		\sigma(S) = \bigcap\limits_{S \subset \MF}\MF, \, \MF \text{ - } \sigma\text{-алгебры}
	$$ 
	называется \uwave{$\sigma$-алгеброй, порожденной $S$}.
\end{defn}
\begin{rem}
	Всегда существуют $\sigma$-алгебры, содержащие $S$, например, $2^X$ обязательно содержит $S$.
\end{rem}
\begin{rem}
	$\sigma(S)$ это \uline{минимальная $\sigma$-алгебра по включению}: если $\sigma$-алгебра $\MA \supset S$, то $\MA \supset \sigma(S)$.
\end{rem}

\textbf{Пример}: Возьмем $S = \{B\} \Rightarrow \sigma(S) = \{\VN,X, B, X \setminus B\}$, мы уже знаем, что это $\sigma$-алгебра.

\begin{rem}
	Построение $\sigma(S)$ это всё, что можно собрать из $S$.
\end{rem}
\begin{exrc}
	Пусть $S = \{B, C\}$, опишите $\sigma(S)$.
\end{exrc}
\begin{proof}
	$$
		\sigma(S) = \{\VN, X, B, C, B \cap C, B \cup C, B \setminus C, C \setminus B, X \setminus B, X \setminus C, 
	$$
	$$
		X \setminus (B \cup C), X \setminus (B \cap C), B \Delta C, X \setminus(B \Delta C), X \setminus (B\setminus C), \, X \setminus ( C \setminus B) \}
	$$
\end{proof}

Когда мы говорим, что ``можно собрать из $S$'' не нужно понимать это буквально, поскольку это не означает, что есть некий алгоритм, который по элементам $S$ с помощью операций: $\cap, \, \cup, \, \setminus$ выводит выражение для любого множества из этой $\sigma$-алгебры. Для больших $S$, $\sigma(S)$ столь огромны, что такого описания нет. Содержательным примером такой $\sigma$-алегбры является Борелевская $\sigma$-алгебра.

\subsection*{Борелевская $\sigma$-алгебра}

\begin{defn}
	Пусть $X$ - метрическое пространство. $\sigma$-алгебра: $\MB(X) = \sigma(\{\text{открытые множества}\})$ называется \uwave{Борелевской} $\sigma$-алгеброй, то есть это минимальная $\sigma$-алгебра, порожденная всеми открытыми множествами $X$.
\end{defn}

\begin{prop}
	$\MB(\MR^n) = \sigma(\{\text{шары}\}) = \sigma(\{\text{открытые кубы}\})$.
\end{prop}
\begin{proof}
	Всякое открытое множество $\MU$ это не более, чем счётное объединение открытых кубов: для каждой точки $a$ строим куб $K_a \subset \MU$ с рациональными вершинами $\Rightarrow \cup_a K_a = \MU$ и таких кубов не более, чем счётное число. Тогда:
	$$
		\MB(\MR^n)\subset \sigma(\{\text{открытые кубы}\})
	$$
	поскольку оно содержит все открытые $\Rightarrow$ должно содержать минимальную порожденную всеми открытми. Обратное включение очевидно, поскольку среди открытых множеств есть открытые кубы.
\end{proof}

\subsection*{$\sigma$-аддитивные меры}

\begin{defn}
	Пусть на $X$ задана $\sigma$-алгебра $\MA$. Функция $\mu \colon \MA \to [0, +\infty)$ называется \uwave{$\sigma$-аддитивной мерой} (конечной неотрицательной $\sigma$-аддитивной мерой), если верно свойство \uwave{$\sigma$-аддитивности}: 
	$$
		\forall A_j \in \MA, \, A_i \cap A_j = \VN, \, \mu\left(\cup_j A_j\right) = \ddsum{j}{}\mu(A_j)
	$$
\end{defn}
\begin{rem}
	Можно допускать в качестве значения $\mu = +\infty$, если добавить в определение:
	\begin{enumerate}[label=\arabic*)]
		\item $\mu(\VN) = 0$;
		\item $\forall c \in \MR, \, c + (+\infty) = + \infty$;
	\end{enumerate}
\end{rem}
\begin{defn}
	Мера $\mu$ называется \uwave{конечной}, если она нигде не принимает значение $+\infty$.
\end{defn}
\begin{rem}
	Заметим, что свойство $\mu(\VN) = 0$ во множествах с конечной мерой появляется автоматически:
	$$
		\mu(X) = \mu(X \cup \VN) = \mu(X) + \mu(\VN) \Rightarrow \mu(\VN) = 0
	$$
\end{rem}

\textbf{Примеры $\sigma$-аддитивных мер}:
\begin{enumerate}[label=\arabic*)]
	\item \uwave{Дельта мера}: Пусть $a \in X$, тогда на $\sigma$-алгебре $2^X$ определена мера:
	$$
		\delta_a(B) = 
		\begin{cases}
			1, & a\in B\\
			0, & a \not\in B
		\end{cases}
	$$
	\begin{proof}
		Если взять объединение попарно непересекающихся множеств, то только одно из них может содержать $a \Rightarrow$ и справа и слева будет $1$, а если ни одно не содержит, то справа и слева будет $0$.
	\end{proof}
	\begin{rem}
		Также эту мерй называют \uwave{мерой Дирака};
	\end{rem}
	\item Пусть $X = \{1,2,\dotsc, N\}$, $\sigma$-алгебра - $2^X$ и заведём числа: $p_k \geq 0$, зададим меру:
	$$
		\mu(B) = p_1{\cdot}\delta_1(B) + \dotsc + p_N{\cdot}\delta_N(B) = \ddsum{k \colon k \in B}{}p_k
	$$
	\begin{proof}
		Слева единички выставятся у тех $p_k$, для которых $k \in B \Rightarrow$ получится нужная нам сумма, а каждая в отдельности $\delta_k$ это $\sigma$-аддитивная мера;
	\end{proof} 
	\item Пусть $X = \MN$, заведем числа: $p_k \geq 0 \colon \sum_k p_k < \infty$, определим меру на $2^\MN$:
	$$
		\mu(B) = \ddsum{k \colon k \in B}{}p_k
	$$
	\begin{exrc}
		Проверить, что эта мера является аддитивной и $\sigma$-аддитивной;
	\end{exrc}
	\item Существует функция $\mu \colon 2^\MN \to [0,+\infty)$ такая, что:
	\begin{enumerate}[label=(\arabic*)]
		\item $\mu$ - аддитивна;
		\item $\mu(\{k\}) = 0$ и $\mu(\MN) = 1$, то есть не является $\sigma$-аддитивной;
	\end{enumerate}
	\begin{rem}
		Для построения такой меры требуются знания из функционального анализа;
	\end{rem}
\end{enumerate}

\newpage
\begin{prop}(\textbf{Непрерывность меры})
	Пусть $\mu$ это конечная $\sigma$-аддитивная мера на $\sigma$-алгебре $\MA$ подмножеств $X$. Тогда:
	\begin{enumerate}[label=\arabic*)]
		\item $A_m \in \MA, \, A_{m+1} \subset A_m \Rightarrow \mu(\cap_m A_m) = \lim\limits_{m \to \infty}\mu(A_m)$;
		\item $B_m \in \MA, \, B_m \subset B_{m+1} \Rightarrow \mu(\cup_m B_m) = \lim\limits_{m \to \infty}\mu(B_m)$;
	\end{enumerate}
\end{prop}
\begin{proof}\hfill
	\begin{enumerate}[label=\arabic*)]
		\item Поймем, что $2) \Rightarrow 1)$: возьмем $X \setminus A_m = B_m$, тогда $X \setminus (\cap_m A_m) = \cup_m (X \setminus A_m) = \cup_m B_m$, заметим:
		$$
			\mu(B_m) = \mu(X) - \mu(A_m)
		$$
		а также, что $B_m \subset B_{m+1} \Rightarrow A_{m+1}\subset A_m$, тогда:
		$$
			\mu(B_m) \leftarrow \mu(\cup_mB_m) = \mu(X) - \mu(\cap_m A_m) \Rightarrow \mu(\cap_m A_m) \to \mu(A_m)
		$$
		\item Рассмотрим множества: $C_1 = B_1, \, C_2 = B_2 \setminus B_1, \, C_3 = B_3 \setminus B_2, \dotsc $, тогда по построению: 
		$$
			B_m = \bigcup\limits_{k = 1}^{m}C_k, \, \forall k,l, \, k \neq l, \,  C_k \cap C_l = \VN \Rightarrow
		$$
		$$
			\Rightarrow \mu(B_m) = \ddsum{k = 1}{m}\mu(C_k) \to \ddsum{k = 1}{\infty}\mu(C_k) = \mu(\cup_k C_k) = \mu(\cup_m B_m)
		$$
		где в предпоследнем равенстве мы воспользовались $\sigma$-аддитивностью;
	\end{enumerate}
\end{proof}

\subsection*{Приближение борелевского множества замкнутыми и открытыми}
\begin{theorem}
	Пусть $X$ - метрическое пространство, $\MB(X)$ - борелевская $\sigma$-алгебра, мера $\mu$ это $\sigma$-аддитивная, конечная мера на $\MB(X)$, тогда: $\forall B \in \MB(X), \, \forall \VE > 0, \, \exists$ замкнутое $F_\VE$, открытое $\MU_\VE$ такие, что:
	\begin{enumerate}[label=\arabic*)]
		\item $F_\VE \subset B \subset \MU_\VE$;
		\item $\mu(\MU_\VE\setminus F_\VE) < \VE$;
	\end{enumerate}
	То есть всякое борелевское множество приближается изнутри и снаружи замкнутым и открытым множеством.
\end{theorem}
\begin{proof}\hfill
	\begin{enumerate}[label=(\arabic*)]
		\item Проверим, что утверждение верно для замкнутых множеств. Пусть $F$ - замкнуто, выберем некоторое  $\VE > 0$, тогда $F = F_\VE$. Рассмотрим открытое множество:
		$$
			\MU_m = F^{\tfrac{1}{m}} = \bigcup\limits_{x \in F}\MB(x,\tfrac{1}{m}) \Rightarrow \MB(x,\tfrac{1}{m+1}) \subset \MB(x,\tfrac{1}{m}) \Rightarrow \MU_{m+1} \subset \MU_m
		$$
		Поскольку $F$ - замкнуто, то $F = \cap_m \MU_m$: если точка не лежит в $F$, то вокруг неё есть шар радиуса $r$ в котором никаких точек из $F$ нет и как только $\tfrac{1}{m} < r$, то эта точка не будет лежать ни в каком шаре $\MB(x,\tfrac{1}{m})$. Тогда мы знаем:
		$$
			\mu(F) = \lim\limits_{m \to \infty}\mu(\MU_m) \Rightarrow \exists \, m \colon \mu(\MU_m \setminus F_\VE) = \mu(\MU_m \setminus F) = \mu(\MU_m) - \mu(F) < \VE
		$$
		\item Проверим верность утверждения для открытых множеств. Рассмотрим набор множеств:
		$$
			S = \{E \in X \colon \forall \VE > 0, \, \exists \, F_\VE, \MU_\VE \colon F_\VE \subset E \subset \MU_\VE \wedge \mu(\MU_\VE \setminus  F_\VE) < \VE \}
		$$
		$S$ содержит все замкнутые множества. Если доказать, что $S$ это $\sigma$-алгебра, то $\MB(x) \subset S$, поскольку борелевская порождается в том числе замкнутыми множествами, а тогда для борелевских множеств автоматически будут выполнены свойства $S$. Докажем это:
		\begin{enumerate}[label=\arabic*)]
			\item $\VN \in S, \, F_\VE = \MU_\VE = \VN$, $X \in S, \, F_\VE = \MU_\VE = X$;
			\item Пусть $E \in S$, возьмем $\VE > 0 \Rightarrow \exists \, F_\VE, \MU_\VE \colon F_\VE \subset E \subset \MU_\VE \Rightarrow X \setminus \MU_\VE \subset X \setminus E \subset X \setminus F_\VE$, тогда:
			$$
				(X \setminus F_\VE) \setminus (X \setminus \MU_\VE) = \MU_\VE \setminus F_\VE \Rightarrow \mu((X \setminus F_\VE) \setminus (X \setminus \MU_\VE)) = \mu(\MU_\VE \setminus F_\VE) < \VE
			$$
			Таким образом, $X \setminus E \in S$;
			\item Остается проверить, замкнутость относительного счетного объединения. Пусть $E_m \in S$, возьмем $\VE > 0$. Рассмотрим множество $E = \cup_m E_m \Rightarrow$ рассмотрим каждое $E_m$:
			$$
				\forall m, \, \exists \, F_m, \MU_m \colon F_m \subset E_m \subset \MU_m \wedge \mu(\MU_m \setminus F_m) < \dfrac{\VE}{2^m}
			$$ 
			Тогда мы можем взять в качестве $\MU = \cup_m \MU_m$: оно будет открытым и будет содержать все  $\MU_m$. Рассмотрим множество $\wte{F} = \cup_m F_m$ оно уже может не быть замкнутым, доработаем его:
			$$
				\wte{F} = \bigcup\limits_m F_m = \bigcup\limits_M F^M, \, F^M = \bigcup\limits_{m = 1}^{M} F_m
			$$
			где $F^M$ уже замкнутые множества, поскольку конечное объединение замкнутых множеств - замкнуто. Мы хотим найти $M$ так, чтобы: $\mu(\MU \setminus F^M) < 2\VE$. Заметим:
			$$
				\lim\limits_{M \to \infty}\mu(\MU \setminus F^M) = \lim\limits_{M \to \infty}\mu(\MU) - \mu(F^M) = \mu(\MU \setminus \wte{F}) \leq \ddsum{m}{}\mu(\MU_m \setminus \wte{F}) \leq
			$$
			$$
				\leq \ddsum{m}{}\mu(\MU_m \setminus F_m)\leq \ddsum{m}{}\dfrac{\VE}{2^m}=\VE \Rightarrow \exists\, M_0 \colon \mu(\MU \setminus F^{M_0}) < 2\VE
			$$
			где мы воспользовались следующим фактом: $\mu(\cup_m \MU_m) \leq \sum_{m}\mu(\MU_m)$, а также тем, что $F_m$ меньше, чем $\wte{F}$. Первое доказывается в курсе действительного анализа, либо так:
			$$
				A_i \cap A_j = \VN \Rightarrow \mu(\cup_m A_m) = \ddsum{m}{}\mu(A_m)
			$$
			по определению для попарно непересекающихся множеств, для любых же будет верно:
			$$
				\mu(B_1 \cup B_2) = \mu(B_1) + \mu(B_2) - \mu(B_1 \cap B_2) \leq \mu(B_1) + \mu(B_2)
			$$
			$$
				\mu(\cup_{j = 1}^{m} B_j) = \mu(\cup_{j =1}^{m-1}B_j \cup B_m) \leq \mu(\cup_{j =1}^{m-1}B_j ) + \mu(B_m)
			$$
			далее в предположении индукции получается для всех. Такие объединения - возрастающая последовательность, переходим к пределу - получаем неравенство. Следовательно, у нас лежат дополнения, объединения $\Rightarrow$ пересечения $\Rightarrow$ у нас $\sigma$-алгебра и  утверждение доказано;
		\end{enumerate}
	\end{enumerate}
\end{proof}

\begin{corollary}
	Если $\mu$ и $\sigma$ - конечные $\sigma$-аддитивные меры на $\MB(X)$ совпадают на всех открытых множествах (тоже самое, что на всех замкнутых), то $\mu = \sigma$ на $\MB(X)$.
\end{corollary}
\begin{proof}
	Пусть $B \in \MB(X)$, $\VE > 0$, $F_\VE \subset B \subset \MU_\VE, \, \mu(\MU_\VE \setminus F_\VE) < \VE$, но поскольку на $B$ меры совпадают, то:
	$$
		\mu(\MU_\VE \setminus F_\VE) = \mu(\MU_\VE) - \mu(F_\VE) = \sigma(\MU_\VE) - \sigma(F_\VE) < \VE \Rightarrow
	$$
	$$
		\Rightarrow \mu(B) - \sigma(B) \leq \mu(\MU_\VE) - \sigma(F_\VE) = \mu(\MU_\VE \setminus F_\VE) <\VE \wedge \sigma(B) - \mu(B) \leq \sigma(\MU_\VE) - \mu(F_\VE) = \sigma(\MU_\VE \setminus F_\VE) <\VE \Rightarrow
	$$
	$$
		\Rightarrow \forall \VE > 0, \, |\sigma(B) - \mu(B)| \leq \VE \Rightarrow \mu(B) = \sigma(B)
	$$
\end{proof}


\end{document}